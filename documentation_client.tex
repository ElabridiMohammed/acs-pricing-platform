\documentclass[12pt,a4paper]{article}
\usepackage[utf8]{inputenc}
\usepackage[T1]{fontenc}
\usepackage[french]{babel}
\usepackage{geometry}
\usepackage{graphicx}
\usepackage{xcolor}
\usepackage{hyperref}
\usepackage{amsmath}
\usepackage{fancyhdr}
\usepackage{array}

\geometry{margin=2.5cm}
\hypersetup{colorlinks=true, linkcolor=blue!70!black, urlcolor=blue!70!black}

\definecolor{cruRed}{HTML}{DC143C}
\definecolor{simBlue}{HTML}{1E90FF}

\pagestyle{fancy}
\fancyhf{}
\rhead{\textit{Documentation Technique}}
\lhead{\textbf{CRU Price Simulator}}
\rfoot{Page \thepage}

\title{
    \vspace{-1cm}
    \textbf{\LARGE Simulateur de Prix CRU} \\[0.3cm]
    \large Application Streamlit pour la Projection Stochastique \\
    des Prix de l'Acide Sulfurique \& du Soufre \\[0.5cm]
    \normalsize \textit{(Parce qu'apparemment, prévoir les prix des commodités,}\\
    \textit{c'est pas assez stressant comme ça)}
}
\author{
    \textbf{Direction Achats Matières Premières} \\
    Département Data \& Analytics \\[0.3cm]
    \small Version 1.0 --- Janvier 2026
}
\date{}

\begin{document}

\maketitle
\thispagestyle{empty}

\vspace{1cm}

\noindent\fbox{\parbox{\textwidth}{
\textbf{Résumé Exécutif} \\[0.2cm]
Ce document présente le \textbf{CRU Commodity Price Simulator}, un outil Python/Streamlit pour simuler des scénarios de prix futurs. L'application intègre de la volatilité stochastique et des chocs de marché (<<~Black Swans~>>) pour générer des projections réalistes.

\textbf{TL;DR} : On prend les prévisions CRU, on ajoute du chaos mathématique calibré, et on obtient des scénarios plus proches de la réalité qu'une simple moyenne.
}}

\newpage
\tableofcontents
\newpage

% ============================================================================
\section{Introduction}
% ============================================================================

Bon, commençons par le commencement. Vous avez probablement reçu ce document parce que quelqu'un vous a dit : \textit{<<~Eh, on a un nouveau truc pour les prévisions de prix, faut que tu regardes~>>}.

L'idée derrière cette application est simple (enfin, relativement) : les prévisions officielles CRU c'est bien, mais ça ne capture pas la \textbf{réalité chaotique} des marchés. Quand une fonderie chilienne décide de faire une maintenance imprévue, les prix ne suivent pas gentiment la courbe prévue --- ils font n'importe quoi pendant quelques mois.

\subsection{Pourquoi cette application ?}

\begin{itemize}
    \item \textbf{Le problème} : Les outlooks CRU donnent une tendance moyenne, pas les scénarios extrêmes
    \item \textbf{La solution} : Simuler des milliers de trajectoires possibles avec Monte Carlo
    \item \textbf{Le bonus} : Pouvoir injecter des <<~spikes~>> (chocs de prix) à volonté
\end{itemize}

\subsection{Ce que l'application fait concrètement}

Elle prend les données Excel du CRU, applique un modèle stochastique, et génère :
\begin{enumerate}
    \item Une \textbf{courbe de base} (l'outlook officiel)
    \item Une \textbf{zone de risque} (intervalle de confiance à 95\%)
    \item Un \textbf{scénario simulé} avec spikes et volatilité
\end{enumerate}

% ============================================================================
\section{Le Modèle Mathématique}
% ============================================================================

\textit{(Cette section contient des maths. Respirez un bon coup.)}

\subsection{Formule Générale}

Le modèle suit l'équation suivante :

\begin{equation}
\boxed{
    P(t) = T(t) \times e^{\sigma \cdot W(t)} \times (1 + S(t))
}
\end{equation}

Où :
\begin{itemize}
    \item $P(t)$ : Prix simulé au mois $t$
    \item $T(t)$ : Tendance interpolée depuis l'Outlook CRU
    \item $\sigma$ : Volatilité annualisée (calibrée sur l'historique ou paramétrable)
    \item $W(t)$ : Mouvement Brownien \textbf{lissé} (avec inertie)
    \item $S(t)$ : Contribution des spikes avec décroissance
\end{itemize}

\subsection{Mouvement Brownien Lissé (AR-1)}

Le problème avec le Brownien classique, c'est que ça donne des courbes qui ressemblent à un électrocardiogramme. Pas très réaliste pour les commodités qui ont de l'\textit{inertie}.

On utilise donc un processus auto-régressif :

\begin{equation}
X(t) = \alpha \cdot X(t-1) + (1-\alpha) \cdot \epsilon(t)
\end{equation}

Avec $\alpha = 0.7$ par défaut (le <<~Curve Smoothing~>> dans l'interface).

\subsection{Spikes avec Décroissance (<<~Shark Fin~>>)}

Quand un spike arrive (genre, la Chine qui ferme ses frontières ou un ouragan sur le Golfe du Mexique), le prix ne redescend pas instantanément. Il y a une \textbf{persistance}.

La décroissance exponentielle :
\begin{equation}
S(t+k) = I \times e^{-3 \cdot k / P}
\end{equation}

Où :
\begin{itemize}
    \item $I$ : Intensité du spike (ex: +50\%)
    \item $P$ : Persistance en mois (ex: 6 mois)
    \item $k$ : Mois depuis le spike
\end{itemize}

\textbf{Résultat visuel} : Une forme d'aileron de requin (<<~Shark Fin~>>) --- montée rapide, descente progressive. Beaucoup plus réaliste qu'un pic vertical qui retombe comme un soufflé raté.

% ============================================================================
\section{Guide d'Utilisation}
% ============================================================================

\subsection{Lancement de l'Application}

\begin{verbatim}
$ cd "/Users/.../domaine project"
$ streamlit run app.py
\end{verbatim}

L'application s'ouvre automatiquement sur \url{http://localhost:8501}.

\subsection{Interface et Paramètres}

\subsubsection{Barre Latérale (Sidebar)}

\begin{center}
\begin{tabular}{|l|l|l|}
\hline
\textbf{Paramètre} & \textbf{Description} & \textbf{Par défaut} \\
\hline
Commodity & Acide Sulfurique ou Soufre & Sulfuric Acid \\
Market / Region & Benchmark géographique & CFR US Gulf \\
Start/End Year & Horizon de projection & 2025-2030 \\
Volatility Mult. & Ajustement de la vol. & 1.0 \\
Spike Frequency & Nombre de spikes/an & 0.5 \\
Spike Intensity & Amplitude du spike (\%) & 30\% \\
Spike Persistence & Durée (mois) & 4 mois \\
Decay Type & Type de décroissance & Exponential \\
Curve Smoothing & Inertie des courbes & 0.70 \\
Monte Carlo Paths & Nombre de simulations & 500 \\
\hline
\end{tabular}
\end{center}

\subsubsection{Zone Principale}

Après avoir cliqué sur \textbf{<<~Run Simulation~>>}, vous obtenez :

\begin{enumerate}
    \item \textbf{Métriques clés} : Prix moyen outlook vs simulé, percentiles, etc.
    \item \textbf{Graphique de projection} avec trois éléments visuels :
    \begin{itemize}
        \item \textcolor{cruRed}{\textbf{Ligne rouge pointillée}} : CRU Outlook (budget)
        \item \textbf{Bande grise} : Zone de risque (IC 95\%)
        \item \textcolor{simBlue}{\textbf{Ligne bleue solide}} : Scénario simulé avec chocs
    \end{itemize}
    \item \textbf{Distributions} : Histogrammes des prix par année
    \item \textbf{Export CSV} : Pour vos analyses Excel
\end{enumerate}

% ============================================================================
\section{Interprétation des Résultats}
% ============================================================================

\subsection{Lecture du Graphique}

\noindent\fbox{\parbox{\textwidth}{
\textbf{Comment lire le graphique :}
\begin{itemize}
    \item La \textbf{ligne rouge pointillée} c'est votre \textit{budget} --- ce que CRU prévoit officiellement
    \item La \textbf{zone grise} c'est le \textit{champ des possibles} --- là où le prix pourrait aller dans 95\% des cas
    \item La \textbf{ligne bleue} c'est un \textit{scénario possible} --- ce qui pourrait arriver si des spikes se produisent
\end{itemize}
}}

\subsection{Exemple de Scénario}

Imaginons les paramètres suivants :
\begin{itemize}
    \item Spike Frequency : 1.5/an (on est pessimistes)
    \item Spike Intensity : 50\% (gros chocs)
    \item Persistence : 6 mois
\end{itemize}

Le graphique montrera des <<~vagues~>> où le prix monte brutalement puis redescend sur 6 mois. C'est ce qu'on observe typiquement lors de :
\begin{itemize}
    \item Arrêts de fonderies (maintenance ou grèves)
    \item Perturbations logistiques (problèmes de fret)
    \item Chocs géopolitiques (sanctions, etc.)
\end{itemize}

% ============================================================================
\section{Limitations et Avertissements}
% ============================================================================

\textit{(Parce qu'aucun modèle n'est parfait, et celui-ci ne fait pas exception)}

\begin{enumerate}
    \item \textbf{Ce n'est pas de la prédiction} : C'est de la simulation stochastique. On génère des scénarios \textit{plausibles}, pas \textit{certains}.
    
    \item \textbf{Les spikes sont aléatoires} : L'application ne prédit pas \textit{quand} un spike va arriver, juste ce qui \textit{pourrait} se passer s'il arrivait.
    
    \item \textbf{La calibration dépend de l'historique} : Si le marché change structurellement, les volatilités historiques peuvent être trompeuses.
    
    \item \textbf{Garbage in, garbage out} : La qualité des résultats dépend de la qualité des données CRU en entrée.
\end{enumerate}

\vspace{0.5cm}
\noindent\fcolorbox{red}{red!10}{\parbox{\textwidth}{
\textbf{Disclaimer :} Cet outil est fourni à titre indicatif pour l'aide à la décision. Il ne constitue en aucun cas une recommandation d'investissement ou une garantie de performance future.

\textit{(Traduction : si vous perdez de l'argent en vous basant uniquement sur ce modèle, c'est pas notre faute)}
}}

% ============================================================================
\section{Support Technique}
% ============================================================================

Si l'application plante, ne s'ouvre pas, ou fait des choses bizarres :

\begin{enumerate}
    \item Vérifier que Python 3.8+ est installé
    \item Vérifier que les dépendances sont installées :
\begin{verbatim}
pip install streamlit pandas numpy plotly scipy openpyxl
\end{verbatim}
    \item Vérifier que le fichier Excel est dans le même dossier
    \item En dernier recours : éteindre et rallumer (oui, ça marche parfois)
\end{enumerate}

% ============================================================================
\section{Conclusion}
% ============================================================================

Voilà, vous avez maintenant un outil qui permet de :
\begin{itemize}
    \item Visualiser les projections CRU avec une couche de réalisme
    \item Tester des scénarios de stress (<<~et si le marché pétait un câble ?~>>)
    \item Avoir des intervalles de confiance pour vos budgets
    \item Impressionner en réunion avec des graphiques interactifs
\end{itemize}

\vspace{0.5cm}

\begin{center}
\textit{<<~Tous les modèles sont faux, mais certains sont utiles.~>>} \\
--- George Box (et probablement votre prof de stats aussi)
\end{center}

\vspace{1cm}

\hrule
\vspace{0.3cm}
\small
\textbf{Document généré automatiquement} --- Pour toute question technique, contacter l'équipe Data \& Analytics.

\end{document}
