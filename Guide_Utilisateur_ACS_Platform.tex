\documentclass[11pt,a4paper]{article}

% ── Packages ──────────────────────────────────────────────────────────────────
\usepackage[utf8]{inputenc}
\usepackage[T1]{fontenc}
\usepackage[french]{babel}
\usepackage{geometry}
\usepackage{graphicx}
\usepackage{xcolor}
\usepackage{titlesec}
\usepackage{enumitem}
\usepackage{booktabs}
\usepackage{tabularx}
\usepackage{array}
\usepackage{fancyhdr}
\usepackage{tcolorbox}
\usepackage{hyperref}
\usepackage{amsmath}
\usepackage{float}
\usepackage{caption}
\usepackage{multicol}
\usepackage{tikz}
\usepackage{setspace}
\usepackage{parskip}
\usepackage{microtype}
\usepackage{tocloft}

\tcbuselibrary{skins,breakable}

% ── Page geometry ─────────────────────────────────────────────────────────────
\geometry{
  top=2.8cm, bottom=2.5cm, left=2.5cm, right=2.5cm,
  headheight=15pt
}

% ── Colours ───────────────────────────────────────────────────────────────────
\definecolor{primary}{HTML}{1E3A5F}
\definecolor{accent}{HTML}{2563EB}
\definecolor{lightbg}{HTML}{F0F4FA}
\definecolor{warningbg}{HTML}{FFF8E1}
\definecolor{warningborder}{HTML}{F59E0B}
\definecolor{tipbg}{HTML}{EBF5FF}
\definecolor{tipborder}{HTML}{2563EB}
\definecolor{darktext}{HTML}{1A2332}
\definecolor{graytext}{HTML}{5A6A85}
\definecolor{lightgray}{HTML}{E2E6ED}
\definecolor{formulabg}{HTML}{F0F4FF}
\definecolor{greenbg}{HTML}{ECFDF5}
\definecolor{greenborder}{HTML}{10B981}
\definecolor{screenbg}{HTML}{F5F5F5}
\definecolor{screenborder}{HTML}{BDBDBD}

% ── Hyperref setup ────────────────────────────────────────────────────────────
\hypersetup{
  colorlinks=true,
  linkcolor=accent,
  urlcolor=accent,
  citecolor=accent,
  pdftitle={Guide Utilisateur -- ACS Pricing Decision Platform},
  pdfsubject={User Guide},
}

% ── Header / Footer ──────────────────────────────────────────────────────────
\pagestyle{fancy}
\fancyhf{}
\renewcommand{\headrulewidth}{0.4pt}
\fancyhead[L]{\small\color{graytext}ACS Pricing Decision Platform}
\fancyhead[R]{\small\color{graytext}Guide Utilisateur}
\fancyfoot[C]{\small\color{graytext}\thepage}

% ── Section styling ──────────────────────────────────────────────────────────
\titleformat{\section}
  {\Large\bfseries\color{primary}}
  {\thesection.}{0.6em}{}
  [\vspace{-0.3em}\textcolor{lightgray}{\rule{\textwidth}{0.8pt}}]

\titleformat{\subsection}
  {\large\bfseries\color{primary}}
  {\thesubsection}{0.5em}{}

\titleformat{\subsubsection}
  {\normalsize\bfseries\color{darktext}}
  {\thesubsubsection}{0.5em}{}

% ── TOC styling ───────────────────────────────────────────────────────────────
\renewcommand{\cftsecfont}{\bfseries\color{primary}}
\renewcommand{\cftsecpagefont}{\bfseries\color{primary}}
\renewcommand{\cftsubsecfont}{\color{darktext}}
\renewcommand{\cftsubsecpagefont}{\color{darktext}}

% ── Custom boxes ──────────────────────────────────────────────────────────────
\newtcolorbox{infobox}[1][]{
  enhanced, breakable,
  colback=tipbg, colframe=tipborder,
  boxrule=0pt, leftrule=3.5pt,
  arc=0pt, outer arc=0pt,
  top=10pt, bottom=10pt, left=14pt, right=14pt,
  fonttitle=\bfseries\color{accent},
  title={#1},
  coltitle=accent,
  attach title to upper={\par\vspace{4pt}},
}

\newtcolorbox{warningbox}[1][]{
  enhanced, breakable,
  colback=warningbg, colframe=warningborder,
  boxrule=0pt, leftrule=3.5pt,
  arc=0pt, outer arc=0pt,
  top=10pt, bottom=10pt, left=14pt, right=14pt,
  fonttitle=\bfseries\color{warningborder},
  title={#1},
  coltitle=warningborder,
  attach title to upper={\par\vspace{4pt}},
}

\newtcolorbox{formulabox}{
  enhanced,
  colback=formulabg, colframe=lightgray,
  boxrule=0.5pt,
  arc=6pt, outer arc=6pt,
  top=12pt, bottom=12pt, left=16pt, right=16pt,
}

\newtcolorbox{examplebox}[1][]{
  enhanced, breakable,
  colback=greenbg, colframe=greenborder,
  boxrule=0pt, leftrule=3.5pt,
  arc=0pt, outer arc=0pt,
  top=10pt, bottom=10pt, left=14pt, right=14pt,
  fonttitle=\bfseries\color{greenborder},
  title={#1},
  coltitle=greenborder,
  attach title to upper={\par\vspace{4pt}},
}

% ── Screenshot placeholder command ───────────────────────────────────────────
\newcommand{\screenshot}[2]{%
  \begin{figure}[H]
    \centering
    \fcolorbox{screenborder}{screenbg}{%
      \parbox{0.88\textwidth}{%
        \vspace{1.8cm}
        \centering
        {\large\color{graytext}\textit{#1}}\\[8pt]
        {\footnotesize\color{graytext}Ins\'erer capture d'\'ecran ici}
        \vspace{1.8cm}
      }%
    }
    \caption{#2}
  \end{figure}
}

% ── Graphics path ─────────────────────────────────────────────────────────────
\graphicspath{{screenshots/}}

% ── Custom commands ───────────────────────────────────────────────────────────
\newcommand{\param}[1]{\texttt{\color{accent}#1}}

% ══════════════════════════════════════════════════════════════════════════════
\begin{document}

% ── COVER PAGE ────────────────────────────────────────────────────────────────
\thispagestyle{empty}
\begin{tikzpicture}[remember picture, overlay]
  \fill[primary] (current page.north west) rectangle
    ([yshift=-4cm]current page.north east);
  \fill[accent] ([yshift=-4cm]current page.north west) rectangle
    ([yshift=-4.15cm]current page.north east);
\end{tikzpicture}

\vspace*{1cm}

\begin{center}
  {\color{white}\fontsize{28}{34}\selectfont\bfseries ACS Pricing}\\[6pt]
  {\color{white}\fontsize{28}{34}\selectfont\bfseries Decision Platform}
\end{center}

\vspace{3.5cm}

\begin{center}
  {\fontsize{22}{28}\selectfont\color{primary}\bfseries Guide Utilisateur}\\[12pt]
  {\large\color{graytext} Version 2.0}
\end{center}

\vspace{3cm}

\begin{center}
\begin{tabular}{rl}
  \textcolor{graytext}{Projet :} & Simulateur de Pricing ACS --- Contrat OCP \\[4pt]
  \textcolor{graytext}{Donn\'ees :} & S\&P Global --- Prix annuels ACS \\[4pt]
  \textcolor{graytext}{Statut :} & Document op\'erationnel
\end{tabular}
\end{center}

\vfill

\begin{center}
  \textcolor{lightgray}{\rule{0.6\textwidth}{0.4pt}}\\[8pt]
  {\small\color{graytext} Document confidentiel --- Usage interne uniquement}
\end{center}

\newpage

% ── TABLE OF CONTENTS ─────────────────────────────────────────────────────────
\thispagestyle{fancy}
\tableofcontents
\newpage

% ══════════════════════════════════════════════════════════════════════════════
% SECTION 1 : INTRODUCTION
% ══════════════════════════════════════════════════════════════════════════════
\section{Introduction}

L'ACS Pricing Decision Platform est un outil interactif d'analyse et de simulation des prix de l'acide sulfurique (ACS) pour les n\'egociations de contrats avec OCP.

L'outil comprend \textbf{trois modules} :

\begin{table}[H]
\centering
\begin{tabularx}{\textwidth}{>{\bfseries\color{primary}}l X}
\toprule
Module & Description \\
\midrule
Monte Carlo Simulation & Projection des prix futurs avec intervalles de confiance et sc\'enarios de spikes. \\[6pt]
Formula Lab \& Decision & Backtest d'une formule contre le march\'e r\'eel, puis test de la formule sur les sc\'enarios Monte Carlo. \\[6pt]
Contract Impact Analysis & Analyse de revenus par volumes, sc\'enarios OCP (70--100\,\%), inflation. \\
\bottomrule
\end{tabularx}
\end{table}

\begin{infobox}[Source des donn\'ees]
Les prix historiques et projections proviennent des \textbf{donn\'ees annuelles S\&P Global}. R\'ef\'erence march\'e : \textbf{ACS CFR North Africa}. Indice variable : \textbf{ACS FOB NW Europe}.
\end{infobox}

\begin{figure}[H]
  \centering
  \includegraphics[width=\textwidth]{01_mc_sidebar.png}
  \caption{Vue g\'en\'erale de la plateforme avec la barre lat\'erale et les onglets.}
\end{figure}


% ══════════════════════════════════════════════════════════════════════════════
% SECTION 2 : MONTE CARLO
% ══════════════════════════════════════════════════════════════════════════════
\section{Monte Carlo Simulation}

Ce module g\'en\`ere des projections de prix par \textbf{simulation Monte Carlo} : \`a partir d'une tendance S\&P et d'une volatilit\'e historique, des centaines de trajectoires possibles sont simul\'ees, avec possibilit\'e d'ajouter des spikes de prix.

\subsection{Param\`etres}

Tous les param\`etres se trouvent dans la \textbf{barre lat\'erale gauche}.

\begin{table}[H]
\centering\small
\begin{tabularx}{\textwidth}{>{\bfseries}l X}
\toprule
Param\`etre & Description \\
\midrule
Product & Variable \`a simuler (ACS CFR North Africa, ACS NW EU, S ME, DAP, etc.) \\[3pt]
Horizon & Ann\'ee de d\'ebut (Start) et de fin (End) de la projection \\[3pt]
Override S\&P & Modifier les perspectives S\&P par ann\'ee (cocher ``Override S\&P values'') \\[3pt]
Petcoke \& Clinker & Override optionnel des perspectives petcoke et clinker (pour F6) \\[3pt]
Volatilit\'e & ``Use Historical'' calcule la volatilit\'e \`a partir des donn\'ees. Multiplier pour l'ajuster. \\[3pt]
Spikes (Frequency) & Nombre de spikes par an (chocs de prix) \\[3pt]
Spikes (Intensity) & Amplitude du spike en \% \\[3pt]
Spikes (Persistence) & Dur\'ee du spike en mois \\[3pt]
Decay Type & D\'ecroissance exponentielle ou lin\'eaire du spike \\[3pt]
MC Paths & Nombre de trajectoires simul\'ees (100 \`a 2\,000) \\[3pt]
Smoothing & Lissage du mouvement brownien (0.3 \`a 0.9) \\[3pt]
Random Seed & Graine pour la reproductibilit\'e des r\'esultats \\
\bottomrule
\end{tabularx}
\end{table}

\begin{figure}[H]
  \centering
  \includegraphics[width=\textwidth]{02_mc_metrics.png}
  \caption{M\'etriques et param\`etres de simulation Monte Carlo.}
\end{figure}

\subsection{R\'esultats}

Apr\`es avoir cliqu\'e sur \textbf{Run Simulation} :

\begin{itemize}[leftmargin=1.5em]
  \item \textbf{5 m\'etriques} en haut --- S\&P Outlook moyen, Simulated Mean, 95th pctl max, 5th pctl min, Volatilit\'e utilis\'ee.
  \item \textbf{Graphe de projection} --- Bande bleue (intervalle 95\,\%), ligne rouge pointill\'ee (tendance S\&P), ligne bleue (sc\'enario individuel s\'electionn\'e via slider).
  \item \textbf{3 histogrammes} --- Distribution des prix pour 3 ann\'ees (fin $-$ 2, fin $-$ 1, fin).
\end{itemize}

\begin{figure}[H]
  \centering
  \includegraphics[width=\textwidth]{03_mc_projection.png}
  \caption{Projection Monte Carlo : bande de confiance 95\,\%, tendance S\&P, sc\'enario individuel.}
\end{figure}

\begin{figure}[H]
  \centering
  \includegraphics[width=\textwidth]{04_mc_distribution.png}
  \caption{Distribution des prix simul\'es pour 3 ann\'ees.}
\end{figure}

\begin{infobox}[Interpr\'etation]
Bande large = forte incertitude. La ligne jaune (moyenne MC, masqu\'ee par d\'efaut) montre la tendance moyenne des simulations. Activer via la l\'egende pour comparer \`a la tendance S\&P.
\end{infobox}


% ══════════════════════════════════════════════════════════════════════════════
% SECTION 3 : FORMULA LAB
% ══════════════════════════════════════════════════════════════════════════════
\section{Formula Lab \& Decision}

Ce module a \textbf{deux fonctions} : (1) tester une formule contre l'historique du march\'e (\textit{backtest}), et (2) appliquer cette formule aux sc\'enarios Monte Carlo pour \'evaluer le risque futur.

\subsection{Les six formules}

L'utilisateur choisit \textbf{une formule \`a la fois} dans le menu d\'eroulant. Chaque formule a ses propres param\`etres ajustables.

\begin{table}[H]
\centering
\begin{tabularx}{\textwidth}{c l X}
\toprule
ID & Nom & Principe \\
\midrule
F1 & Sulfur Indexing & Indexation directe soufre (ME/NA) + co\^ut de production \\[3pt]
F2 & Smooth Sulfur & Comme F1, mais avec moyenne liss\'ee du soufre \\[3pt]
F3 & Last Month ACS & Prix ACS du mois pr\'ec\'edent (pond\'eration r\'egionale) \\[3pt]
F4 & S \& DAP Variation & Variation soufre + DAP par rapport \`a des r\'ef\'erences \\[3pt]
F5 & Smooth S \& DAP & Comme F4, avec inputs liss\'es \\[3pt]
F6 & Full Cost Stack & Soufre + DAP + petcoke + clinker (4 composantes) \\
\bottomrule
\end{tabularx}
\end{table}

\subsection{Param\`etres de chaque formule}

Chaque formule expose ses coefficients (a, b, c\ldots), ses poids r\'egionaux, et ses prix de r\'ef\'erence. Un \textbf{Floor} (plancher) et un \textbf{Cap} (plafond) peuvent \^etre activ\'es ou d\'esactiv\'es via un toggle.

\begin{figure}[H]
  \centering
  \includegraphics[width=\textwidth]{05_formula_params.png}
  \caption{S\'election de formule, \'equation, param\`etres ajustables et toggle Floor/Cap.}
\end{figure}

\subsection{Backtest --- Historique}

Le backtest compare la formule choisie au march\'e r\'eel sur l'historique (2018--2025 par d\'efaut, ajustable via slider).

\begin{itemize}[leftmargin=1.5em]
  \item \textbf{Graphe Formula vs Market} --- Deux courbes : prix march\'e (rouge) et prix formule (bleu). Lignes Floor/Cap si activ\'ees.
  \item \textbf{Graphe P\&L} --- Barres vertes (formule sous le march\'e = avantage acheteur) ou rouges (formule au-dessus = d\'esavantage). Vue trimestrielle ou annuelle.
\end{itemize}

\begin{figure}[H]
  \centering
  \includegraphics[width=\textwidth]{06_formula_backtest.png}
  \caption{Backtest : prix march\'e (rouge) vs prix formule (bleu) sur l'historique.}
\end{figure}

\begin{figure}[H]
  \centering
  \includegraphics[width=\textwidth]{07_formula_pnl.png}
  \caption{P\&L historique : barres vertes/rouges par trimestre avec moyenne.}
\end{figure}

\subsection{Performance Summary}

Cinq cartes r\'esument le backtest :

\begin{table}[H]
\centering
\begin{tabularx}{\textwidth}{>{\bfseries}l X}
\toprule
Carte & Description \\
\midrule
Avg P\&L & \'Ecart moyen formule vs march\'e (\$/t) \\[3pt]
Win Rate & \% des p\'eriodes o\`u la formule est sous le march\'e \\[3pt]
Best Period & Meilleur \'ecart en faveur de l'acheteur \\[3pt]
Worst Period & Pire \'ecart (formule au-dessus du march\'e) \\[3pt]
Annual Impact & Impact financier annualis\'e en \$M (750\,KT) \\
\bottomrule
\end{tabularx}
\end{table}

\begin{figure}[H]
  \centering
  \includegraphics[width=\textwidth]{12_formula_performance.png}
  \caption{Cartes Performance Summary : Avg P\&L, Win Rate, Best, Worst, Annual Impact.}
\end{figure}

\subsection{Decision --- Test sur sc\'enarios Monte Carlo}

\begin{warningbox}[Pr\'erequis]
Il faut d'abord lancer une simulation Monte Carlo dans l'\textbf{onglet 1} pour g\'en\'erer des sc\'enarios futurs. Ensuite, revenir ici pour appliquer la formule choisie.
\end{warningbox}

Cette section applique la formule s\'electionn\'ee \`a un sc\'enario Monte Carlo pour \'evaluer son comportement futur.

\begin{itemize}[leftmargin=1.5em]
  \item \textbf{Annual Volume} --- Volume annuel (d\'efaut 750\,KT) pour le calcul financier.
  \item \textbf{Sc\'enario MC} --- Slider pour choisir un sc\'enario parmi ceux g\'en\'er\'es.
  \item \textbf{Graphe Formula vs MC} --- Formule appliqu\'ee au sc\'enario futur (prix march\'e simul\'e vs prix formule).
  \item \textbf{P\&L futur} --- Barres vertes/rouges sur la p\'eriode simul\'ee.
  \item \textbf{3 cartes} --- Market Cost, Formula Cost, Savings (\$M/yr).
\end{itemize}

\begin{figure}[H]
  \centering
  \includegraphics[width=\textwidth]{08_formula_decision.png}
  \caption{Formule appliqu\'ee \`a un sc\'enario Monte Carlo futur.}
\end{figure}

\subsection{Risk Analysis --- Tous les sc\'enarios MC}

L'outil calcule automatiquement les savings de la formule sur \textbf{tous les sc\'enarios MC} (jusqu'\`a 200) et affiche :

\begin{itemize}[leftmargin=1.5em]
  \item \textbf{Expected Savings} --- Savings moyen sur tous les sc\'enarios.
  \item \textbf{Best Case (P95)} --- Savings dans le sc\'enario le plus favorable.
  \item \textbf{Worst Case (P5)} --- Savings dans le sc\'enario le plus d\'efavorable.
  \item \textbf{Prob. of Savings} --- \% des sc\'enarios qui montrent des \'economies.
  \item \textbf{Histogramme} --- Distribution des savings sur l'ensemble des sc\'enarios.
\end{itemize}

\begin{figure}[H]
  \centering
  \includegraphics[width=\textwidth]{09_formula_risk.png}
  \caption{Risk Analysis : cartes Expected/Best/Worst/Probability + histogramme des savings.}
\end{figure}

\begin{infobox}[Aide \`a la d\'ecision]
Si la ``Prob. of Savings'' est $> 50$\,\% et le ``Worst Case'' reste acceptable, la formule offre une bonne protection. Comparer plusieurs formules en changeant la s\'election dans le menu d\'eroulant.
\end{infobox}


% ══════════════════════════════════════════════════════════════════════════════
% SECTION 4 : CONTRACT IMPACT ANALYSIS
% ══════════════════════════════════════════════════════════════════════════════
\section{Contract Impact Analysis}

C'est le module principal pour l'analyse contractuelle. Il mod\'elise les revenus projet et \'evalue l'impact OCP sur la dur\'ee du contrat.

\subsection{Param\`etres}

Les param\`etres sont affich\'es \textbf{en haut de l'onglet} (pas dans la barre lat\'erale), en trois colonnes.

\begin{table}[H]
\centering
\begin{tabularx}{\textwidth}{>{\bfseries}l c X}
\toprule
Param\`etre & D\'efaut & Description \\
\midrule
Total Production Volume & 750 KT & Capacit\'e annuelle de production \\[4pt]
Fixed Price & 110 \$/t & Prix fixe contractuel (ann\'ee de base) \\[4pt]
OCP Purchase Volume & 70--100\,\% & Part achet\'ee par OCP. 70\,\% = tout en fixe \\[4pt]
Inflation Rate & 2\,\% & Taux annuel d'inflation sur le prix fixe \\[4pt]
Coefficient A & 1.0 & Multiplicateur FOB NW Europe \\[4pt]
Premium B & 0 \$/t & Prime/remise ajout\'ee au prix variable \\
\bottomrule
\end{tabularx}
\end{table}

Sous les param\`etres, trois \textbf{bo\^ites de formule} affichent en temps r\'eel le prix fixe, le prix variable, et la r\'epartition des volumes.

\begin{figure}[H]
  \centering
  \includegraphics[width=\textwidth]{10_contract_params.png}
  \caption{Param\`etres Contract Impact : 3 colonnes + bo\^ites de formules.}
\end{figure}

\begin{warningbox}[Logique des volumes]
OCP est oblig\'e d'acheter au minimum \textbf{70\,\% de la production} au \textbf{prix fixe}.
Le curseur va de 70\,\% \`a 100\,\%. Tout volume au-dessus de 70\,\% est achet\'e au \textbf{prix variable} (formule $A \times \text{FOB}_{\text{NWE}} + B$).

\begin{itemize}[leftmargin=1.2em]
  \item \`A 70\,\% : OCP ach\`ete 525\,KT, tout au prix fixe.
  \item \`A 85\,\% : OCP ach\`ete 637\,KT = 525\,KT fixe + 112\,KT variable.
  \item \`A 100\,\% : OCP ach\`ete 750\,KT = 525\,KT fixe + 225\,KT variable.
\end{itemize}
\end{warningbox}

\subsection{Formules}

\begin{formulabox}
\textbf{Prix Fixe (avec inflation) :}
\[
P_{\text{fixe}}(t) = P_{\text{base}} \times (1 + i)^{t - t_0}
\]
o\`u $i$ = taux d'inflation annuel, $t_0$ = premi\`ere ann\'ee du filtre.

\textbf{Prix Variable :}
\[
P_{\text{var}} = A \times \text{FOB}_{\text{NW Europe}} + B
\]
\end{formulabox}

\subsection{Year Range \& Override march\'e}

\begin{itemize}[leftmargin=1.5em]
  \item \textbf{Year Range} --- Slider pour restreindre la p\'eriode d'analyse (ex. 2022--2035).
  \item \textbf{Override market price projections} --- Cocher pour saisir des prix march\'e personnalis\'es par ann\'ee. Tous les graphiques se recalculent automatiquement.
\end{itemize}

\begin{figure}[H]
  \centering
  \includegraphics[width=\textwidth]{13_contract_override.png}
  \caption{Override des prix march\'e : champs par ann\'ee pour modifier les prix ACS CFR North Africa.}
\end{figure}

\subsection{Perspective Selector}

Un bouton radio permet de basculer entre deux vues : \textbf{Project Perspective} (vendeur) et \textbf{OCP Perspective} (acheteur).

\subsection{Project Perspective}

Point de vue du vendeur : combien le projet g\'en\`ere de revenus.

\begin{itemize}[leftmargin=1.5em]
  \item \textbf{5 m\'etriques} --- Avg Revenue, Weighted Avg Price, Market Ref Price, Negotiated Price, vs Market (\%).
  \item \textbf{Business Plan Revenue} --- Barres empil\'ees : revenu fixe (bleu) + revenu variable (jaune) par ann\'ee.
  \item \textbf{Volume Split} --- Barre empil\'ee montrant la r\'epartition fixe/variable en KT.
  \item \textbf{Price Comparison} --- 5 courbes : Market Ref (noir), Weighted Avg (jaune), Fixed Price avec inflation (bleu pointill\'e), Negotiated Price (rouge pointill\'e), FOB NW Europe (violet).
  \item \textbf{Sensitivity} --- Tableau comparant diff\'erents splits de volume (50/50, 60/40, \ldots, 100/0) et leur impact sur le revenu moyen.
\end{itemize}

\begin{figure}[H]
  \centering
  \includegraphics[width=\textwidth]{14_project_revenue.png}
  \caption{Business Plan Revenue : barres empil\'ees revenu fixe (bleu) et variable (jaune) par ann\'ee.}
\end{figure}

\begin{figure}[H]
  \centering
  \includegraphics[width=\textwidth]{15_project_prices.png}
  \caption{Price Comparison : 5 courbes de prix (march\'e, pond\'er\'e, fixe, n\'egoci\'e, FOB NWE).}
\end{figure}

\subsection{OCP Perspective}

Point de vue de l'acheteur : combien OCP \'economise par rapport au march\'e.

\begin{itemize}[leftmargin=1.5em]
  \item \textbf{4 m\'etriques} --- OCP Avg Cost, Market Cost, Avg Value Gain, Avg Price Paid.
  \item \textbf{OCP Value Gain} --- Barres par ann\'ee. Bleu = gain portion fixe, jaune = gain portion variable. Les barres n\'egatives descendent sous z\'ero (barmode ``relative'').
  \item \textbf{OCP Price Paid vs Market} --- 3 sc\'enarios pr\'ed\'efinis (70\,\% fixe, 85\,\%, 100\,\%) + march\'e.
  \item \textbf{Cumulative Value Gain} --- Gain cumul\'e dans le temps pour les 3 sc\'enarios.
  \item \textbf{Breakeven Analysis} --- March\'e (noir) vs blended contract (jaune) vs fixe avec inflation (bleu pointill\'e). Zone omhr\'ee = zone de savings.
  \item \textbf{Scenario Comparison Summary} --- Tableau r\'ecapitulatif : volume, co\^ut, gain pour chaque sc\'enario.
\end{itemize}

\begin{figure}[H]
  \centering
  \includegraphics[width=\textwidth]{11_contract_charts.png}
  \caption{OCP Value Gain : barres de gain/perte annuel par composante fixe et variable.}
\end{figure}

\begin{figure}[H]
  \centering
  \includegraphics[width=\textwidth]{16_ocp_prices.png}
  \caption{OCP Price Paid vs Market : 3 sc\'enarios (70\,\%, 85\,\%, 100\,\% fixe) compar\'es au march\'e.}
\end{figure}

\begin{figure}[H]
  \centering
  \includegraphics[width=\textwidth]{17_breakeven.png}
  \caption{Breakeven Analysis : march\'e vs contrat avec zone de savings.}
\end{figure}

\begin{figure}[H]
  \centering
  \includegraphics[width=\textwidth]{18_scenario_summary.png}
  \caption{Scenario Comparison Summary : tableau r\'ecapitulatif des 3 sc\'enarios OCP.}
\end{figure}

\begin{warningbox}[Point cl\'e]
Quand le march\'e est \'elev\'e, le contrat \`a prix fixe avantage OCP (gain positif). Quand le march\'e baisse sous le prix fixe, le gain devient n\'egatif. L'inflation augmente le prix fixe chaque ann\'ee, ce qui r\'eduit progressivement l'avantage.
\end{warningbox}


% ══════════════════════════════════════════════════════════════════════════════
% SECTION 5 : FONCTIONNALITES COMMUNES
% ══════════════════════════════════════════════════════════════════════════════
\section{Fonctionnalit\'es g\'en\'erales}

\begin{itemize}[leftmargin=1.5em, itemsep=6pt]
  \item \textbf{Recalcul automatique} --- Tous les graphiques se mettent \`a jour d\`es qu'un param\`etre change.
  \item \textbf{Filtrage temporel} --- Sliders ``Year Range'' ou ``History from'' pour restreindre la p\'eriode.
  \item \textbf{Export graphiques} --- Survoler un graphe, cliquer sur l'ic\^one appareil photo pour sauvegarder en PNG.
  \item \textbf{Export donn\'ees} --- Ouvrir les \'el\'ements ``Detailed Data'' pour copier-coller dans Excel.
  \item \textbf{L\'egendes interactives} --- Cliquer sur une s\'erie dans la l\'egende pour la masquer/afficher.
\end{itemize}


% ══════════════════════════════════════════════════════════════════════════════
% SECTION 6 : GLOSSAIRE
% ══════════════════════════════════════════════════════════════════════════════
\section{Glossaire}

\begin{table}[H]
\centering
\begin{tabularx}{\textwidth}{>{\bfseries}l X}
\toprule
Terme & D\'efinition \\
\midrule
ACS & Acide sulfurique \\[3pt]
CFR North Africa & Prix march\'e ACS livr\'e Afrique du Nord (r\'ef\'erence) \\[3pt]
FOB NW Europe & Prix ACS Nord-Ouest Europe (indice variable) \\[3pt]
OCP & Office Ch\'erifien des Phosphates \\[3pt]
KT & Kilo-tonne (1\,000 tonnes) \\[3pt]
Value Gain & \'Economie OCP = (Prix march\'e $-$ Prix pay\'e) $\times$ Volume \\[3pt]
Breakeven & Point o\`u le prix contractuel \'egale le prix march\'e \\[3pt]
P\&L & Profit and Loss = March\'e $-$ Formule (\$/t) \\[3pt]
Monte Carlo & Simulation statistique multi-sc\'enarios \\[3pt]
S\&P & S\&P Global Commodity Insights (source des donn\'ees) \\[3pt]
DAP & Phosphate di-ammonique \\[3pt]
Floor / Cap & Plancher et plafond impos\'es sur le prix formule \\
\bottomrule
\end{tabularx}
\end{table}


% ══════════════════════════════════════════════════════════════════════════════
% SECTION 7 : FAQ
% ══════════════════════════════════════════════════════════════════════════════
\section{Questions fr\'equentes}

\begin{description}[leftmargin=0em, style=nextline, font=\normalfont\bfseries\color{primary}]

\item[Que signifie un Value Gain n\'egatif ?]
OCP paie plus cher que le march\'e. Cela arrive quand le march\'e baisse sous le prix fixe.

\item[Pourquoi le curseur OCP va uniquement de 70 \`a 100\,\% ?]
OCP est contractuellement oblig\'e d'acheter au minimum 70\,\% au prix fixe. Le reste (0 \`a 30\,\%) est optionnel et achet\'e au prix variable.

\item[Comment simuler une hausse du march\'e ?]
Dans l'onglet Contract Impact, cocher ``Override market price projections'' et saisir des valeurs plus \'elev\'ees. Tous les graphiques se mettent \`a jour.

\item[Comment fonctionne l'inflation ?]
Le prix fixe augmente chaque ann\'ee : $110 \times (1{,}02)^n$. Ann\'ee 2 = 112,2\,\$/t, ann\'ee 5 = 121,4\,\$/t, etc.

\item[Pourquoi faut-il lancer Monte Carlo avant d'utiliser ``Decision'' ?]
La section Decision dans l'onglet Formula Lab applique la formule aux sc\'enarios MC. Sans simulation, il n'y a pas de sc\'enarios \`a tester.

\item[Comment comparer deux formules ?]
Tester la premi\`ere formule (noter les r\'esultats), puis changer la s\'election dans le menu d\'eroulant et comparer. Il n'y a pas de tableau comparatif automatique --- la comparaison se fait formule par formule.

\item[Comment exporter un graphique ?]
Survoler le graphique et cliquer sur l'ic\^one appareil photo en haut \`a droite.

\item[Que fait le bouton ``Enforce Floor / Cap'' ?]
Quand activ\'e, le prix formule ne peut pas descendre sous le Floor ni d\'epasser le Cap. D\'esactiv\'e, la formule est libre (les lignes sont affich\'ees en gris comme r\'ef\'erence).

\end{description}


\vfill

\begin{center}
\textcolor{lightgray}{\rule{0.5\textwidth}{0.4pt}}\\[8pt]
{\small\color{graytext} ACS Pricing Decision Platform --- Guide Utilisateur v2.0}\\[3pt]
{\small\color{graytext} Document confidentiel --- Usage interne uniquement}
\end{center}

\end{document}
