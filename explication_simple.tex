\documentclass[11pt,a4paper]{article}
\usepackage[utf8]{inputenc}
\usepackage[T1]{fontenc}
\usepackage[french]{babel}
\usepackage{geometry}
\usepackage{xcolor}
\usepackage{amsmath}

\geometry{margin=2cm, top=1.5cm, bottom=1.5cm}
\setlength{\parindent}{0pt}
\setlength{\parskip}{6pt}
\pagestyle{empty}

\definecolor{accent}{HTML}{1E90FF}

\begin{document}

\begin{center}
{\LARGE \textbf{Simulateur de Prix CRU}} \\[0.2cm]
{\large Comment ça marche ? (version humaine)} \\[0.3cm]
\small Par Mohammed ELARIDI --- Janvier 2026
\end{center}
\hrule
\vspace{0.4cm}

\textbf{\large Le Problème}

CRU dit : <<~140\$/t en 2028~>>. Cool. Sauf que le marché s'en fiche des prévisions. Entre grèves, ouragans et décisions politiques random, le prix fait rarement ce qu'on lui dit.

\textbf{Solution :} On simule PLEIN de scénarios au lieu de croire à UN seul chiffre.

\vspace{0.3cm}
\textbf{\large La Formule (Promis, c'est la seule)}

\begin{center}
\fcolorbox{accent}{blue!5}{
$\textbf{Prix}(t) = \underbrace{\text{Tendance CRU}}_{\text{ce que dit CRU}} \times \underbrace{e^{\text{bruit}}}_{\text{la vraie vie}} \times \underbrace{(1 + \text{spike})}_{\text{les catastrophes}}$
}
\end{center}

\begin{itemize} \setlength{\itemsep}{0pt}
    \item \textbf{Tendance CRU} = La prévision officielle (ligne rouge pointillée)
    \item \textbf{Le bruit} = Variations quotidiennes : +2\% un mois, -3\% le suivant
    \item \textbf{Les spikes} = Quand quelqu'un fait grève ou qu'un ouragan passe
\end{itemize}

\vspace{0.3cm}
\textbf{\large Le Bruit (Volatilité)}

Imaginez que le prix essaie de suivre CRU, mais qu'il est un peu bourré. Il zigzague.

\textbf{Problème :} Bruit 100\% aléatoire = électrocardiogramme. Pas réaliste.

\textbf{Solution :} On ajoute de la <<~mémoire~>>. Si ça montait hier, ça continue un peu aujourd'hui.
$$\text{Mouvement}(t) = 0.7 \times \text{Mouvement}(t-1) + 0.3 \times \text{Choc}$$

Le 0.7 c'est le <<~smoothing~>> --- plus c'est haut, plus c'est lisse.

\vspace{0.3cm}
\textbf{\large Les Spikes (Chocs de Marché)}

Parfois il se passe des trucs. Une fonderie ferme, un pays impose des sanctions : le prix explose.

\textbf{Forme :} <<~Aileron de requin~>> --- montée rapide (1 mois), descente progressive (3-6 mois).
$$\text{Spike}(t+k) = \text{Intensité} \times e^{-3k / \text{Persistance}}$$
En gros : le choc s'estompe exponentiellement. Comme un souvenir douloureux.

\vspace{0.3cm}
\textbf{\large Monte Carlo : 500 Simulations}

On lance 500 simulations avec des dés différents à chaque fois :
\begin{itemize} \setlength{\itemsep}{0pt}
    \item Simulation \#1 : Pas de spike $\rightarrow$ prix stable
    \item Simulation \#47 : Gros spike en 2027 $\rightarrow$ +50\%
    \item Simulation \#233 : Deux spikes $\rightarrow$ chaos
\end{itemize}
$\rightarrow$ La \textbf{bande grise} = 95\% des scénarios possibles.

\vspace{0.3cm}
\textbf{\large Le Graphique}

\begin{center}
\begin{tabular}{|c|l|}
\hline
\textcolor{red}{\textbf{Rouge pointillé}} & Prévision CRU (budget) \\
\textcolor{gray}{\textbf{Bande grise}} & Zone de risque (95\%) \\
\textcolor{blue}{\textbf{Bleu solide}} & Scénario simulé \\
\hline
\end{tabular}
\end{center}

\vspace{0.3cm}
\textbf{\large En Résumé}

\begin{enumerate} \setlength{\itemsep}{0pt}
    \item Prévision CRU + bruit lissé + spikes <<~requin~>>
    \item $\times$ 500 simulations
    \item = Plage de prix possibles
\end{enumerate}

\vspace{0.4cm}
\begin{center}
\fcolorbox{black}{yellow!20}{
\parbox{0.9\textwidth}{\small
\textbf{Disclaimer :} Ce n'est pas de la prédiction. On ne sait pas CE QUI VA se passer, on sait CE QUI POURRAIT se passer.
}}
\end{center}

\vspace{0.3cm}
\hrule
\vspace{0.2cm}
\small \textbf{Contact :} Mohammed ELARIDI \hfill \textit{<<~Tous les modèles sont faux, mais certains sont utiles.~>>}

\end{document}
