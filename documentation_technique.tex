\documentclass[12pt,a4paper]{article}
\usepackage[utf8]{inputenc}
\usepackage[T1]{fontenc}
\usepackage[french]{babel}
\usepackage{geometry}
\usepackage{graphicx}
\usepackage{xcolor}
\usepackage{hyperref}
\usepackage{amsmath}
\usepackage{amssymb}
\usepackage{fancyhdr}
\usepackage{array}
\usepackage{booktabs}

\geometry{margin=2.5cm}
\hypersetup{colorlinks=true, linkcolor=blue!70!black, urlcolor=blue!70!black}

\definecolor{codeblue}{HTML}{1E90FF}
\definecolor{mathgreen}{HTML}{2E8B57}

\pagestyle{fancy}
\fancyhf{}
\rhead{\textit{Documentation Technique Simplifiée}}
\lhead{\textbf{CRU Price Simulator}}
\rfoot{Page \thepage}

\setlength{\headheight}{14.5pt}

\title{
    \vspace{-1cm}
    \textbf{\LARGE Simulateur de Prix CRU} \\[0.5cm]
    \Large Guide Technique Simplifié \\[0.3cm]
    \large Mathématiques, Code \& Interface \\[0.5cm]
    \normalsize \textit{Expliqué comme si vous aviez 5 ans}\\
    \textit{(bon, peut-être 15 ans avec des notions de stats)}
}
\author{
    \textbf{Département Data \& Analytics} \\[0.3cm]
    \small Version 2.0 --- Janvier 2026
}
\date{}

\begin{document}

\maketitle
\thispagestyle{empty}

\vspace{0.5cm}

\noindent\fbox{\parbox{\textwidth}{
\textbf{Ce document explique :}
\begin{enumerate}
    \item Comment le modèle mathématique fonctionne (en termes simples)
    \item Ce que fait chaque ligne de code importante
    \item À quoi sert chaque slider/paramètre dans l'interface
\end{enumerate}
}}

\newpage
\tableofcontents
\newpage

% ============================================================================
\section{L'Idée Générale (En 30 Secondes)}
% ============================================================================

Imaginez que vous voulez savoir combien coûtera l'acide sulfurique dans 3 ans.

\textbf{Option 1 :} Regarder la prévision CRU $\rightarrow$ <<~140 \$/t~>> \\
\textbf{Problème :} Le marché ne suit jamais exactement les prévisions.

\textbf{Option 2 (ce qu'on fait) :} 
\begin{enumerate}
    \item Prendre la prévision CRU comme <<~tendance de base~>>
    \item Ajouter du <<~bruit~>> aléatoire (volatilité du marché)
    \item Ajouter des <<~chocs~>> possibles (spikes)
    \item Répéter 500 fois pour voir toutes les possibilités
\end{enumerate}

\textbf{Résultat :} Au lieu d'UN chiffre, on a une PLAGE de prix possibles.

% ============================================================================
\section{Le Modèle Mathématique}
% ============================================================================

\subsection{La Formule Magique}

Tout le simulateur repose sur UNE formule :

\begin{equation}
\boxed{
    \textcolor{codeblue}{Prix(t)} = \textcolor{mathgreen}{Tendance(t)} \times e^{\sigma \cdot W(t)} \times (1 + Spike(t))
}
\end{equation}

Décomposons ça :

\begin{center}
\begin{tabular}{|c|p{10cm}|}
\hline
\textbf{Terme} & \textbf{Signification en français} \\
\hline
$Prix(t)$ & Le prix simulé au mois $t$ \\
\hline
$Tendance(t)$ & La prévision CRU interpolée (la ligne rouge pointillée) \\
\hline
$e^{\sigma \cdot W(t)}$ & Le facteur de volatilité : fait varier le prix autour de la tendance \\
\hline
$(1 + Spike(t))$ & Le multiplicateur de spike : +30\% si choc, sinon $\times 1$ \\
\hline
\end{tabular}
\end{center}

\subsection{Composant 1 : La Tendance $T(t)$}

\textbf{D'où ça vient ?} Du fichier Excel CRU Outlook.

\textbf{Le problème :} Les données sont annuelles (2025, 2026, 2027...) \\
\textbf{La solution :} On interpole avec une courbe lisse (spline cubique)

\textbf{En code Python :}
\begin{verbatim}
from scipy.interpolate import interp1d
f = interp1d(years, prices, kind='cubic')
monthly_trend = f(monthly_dates)
\end{verbatim}

\textbf{Résultat :} On passe de 6 points annuels à 72 points mensuels.

\subsection{Composant 2 : La Volatilité $\sigma \cdot W(t)$}

\textbf{C'est quoi ?} Du <<~bruit~>> aléatoire qui fait bouger le prix.

\textbf{Le problème initial :} Le bruit classique (Brownien simple) fait des courbes <<~en dents de scie~>> --- pas réaliste.

\textbf{La solution :} On <<~lisse~>> le bruit avec un processus AR(1) :

\begin{equation}
X(t) = \underbrace{\alpha \cdot X(t-1)}_{\text{mémoire du passé}} + \underbrace{(1-\alpha) \cdot \epsilon(t)}_{\text{nouveau choc}}
\end{equation}

\begin{itemize}
    \item $\alpha$ = facteur de lissage (0.7 par défaut)
    \item Plus $\alpha$ est grand, plus la courbe est <<~douce~>>
    \item $\epsilon(t)$ = choc aléatoire normal $\sim N(0, \sigma)$
\end{itemize}

\textbf{En code Python :}
\begin{verbatim}
smoothed[0] = raw_shocks[0]
for t in range(1, n_steps):
    smoothed[t] = alpha * smoothed[t-1] + (1-alpha) * raw_shocks[t]
\end{verbatim}

\textbf{En image :}
\begin{center}
Sans lissage : $\sim\sim\sim\sim$ (electrocardiogramme) \\
Avec lissage : $\smile\frown\smile$ (vagues douces)
\end{center}

\subsection{Composant 3 : Les Spikes $S(t)$}

\textbf{C'est quoi ?} Des chocs de prix soudains (grèves, ouragans, sanctions...).

\textbf{Comment ça marche ?}
\begin{enumerate}
    \item On tire au hasard si un spike arrive (probabilité = fréquence/12)
    \item Si spike : le prix monte de $I$\% (intensité)
    \item Puis le prix redescend progressivement sur $P$ mois (persistance)
\end{enumerate}

\textbf{La formule de décroissance :}

\begin{equation}
Spike(t+k) = I \times e^{-3 \cdot k / P}
\end{equation}

\begin{center}
\begin{tabular}{|c|l|}
\hline
$k=0$ (jour du spike) & $Spike = I$ (ex: +50\%) \\
$k=P/3$ & $Spike \approx 0.37 \times I$ (+18\%) \\
$k=P$ (fin de persistance) & $Spike \approx 0.05 \times I$ (+2.5\%) \\
\hline
\end{tabular}
\end{center}

\textbf{Forme visuelle :} <<~Aileron de requin~>> (Shark Fin) --- montée brutale, descente progressive.

\textbf{En code Python :}
\begin{verbatim}
for k in range(persistence):
    progress = k / persistence
    decay = exp(-3.0 * progress)  # Decroissance exponentielle
    spike_contribution[t + k] += intensity * decay
\end{verbatim}

\subsection{Composant 4 : Monte Carlo (500 simulations)}

\textbf{Pourquoi 500 ?} Pour avoir assez de scénarios et calculer des statistiques fiables.

\textbf{Ce qu'on calcule :}
\begin{itemize}
    \item \textbf{Moyenne} : le scénario <<~central~>>
    \item \textbf{Percentile 5\%} : pire cas (prix bas)
    \item \textbf{Percentile 95\%} : pire cas (prix haut)
\end{itemize}

\textbf{En code Python :}
\begin{verbatim}
all_paths = np.zeros((500, n_months))  # 500 simulations x 72 mois

for i in range(500):
    brownian = generate_brownian_motion(...)
    spikes = generate_spike_process(...)
    all_paths[i] = trend * exp(brownian) * (1 + spikes)

mean = np.mean(all_paths, axis=0)
p5 = np.percentile(all_paths, 5, axis=0)
p95 = np.percentile(all_paths, 95, axis=0)
\end{verbatim}

% ============================================================================
\section{Guide Complet des Paramètres de l'Interface}
% ============================================================================

Voici TOUS les sliders et options de l'application, avec leur effet exact.

\subsection{Section : Commodity Selection}

\begin{center}
\begin{tabular}{|p{4cm}|p{3cm}|p{6cm}|}
\hline
\textbf{Paramètre} & \textbf{Valeurs} & \textbf{Ce que ça change} \\
\hline
\texttt{Commodity} & Sulfuric Acid / Sulphur & Change le fichier source (feuille Excel différente) \\
\hline
\texttt{Market / Region} & CFR US Gulf, FOB Japan... & Sélectionne la colonne de prix dans l'Excel \\
\hline
\end{tabular}
\end{center}

\subsection{Section : Projection Horizon}

\begin{center}
\begin{tabular}{|p{4cm}|p{3cm}|p{6cm}|}
\hline
\textbf{Paramètre} & \textbf{Valeurs} & \textbf{Ce que ça change} \\
\hline
\texttt{Start Year} & 2025, 2026, 2027 & Année de début de la simulation \\
\hline
\texttt{End Year} & 2028, 2029, 2030 & Année de fin $\rightarrow$ détermine le nombre de mois \\
\hline
\end{tabular}
\end{center}

\textbf{Exemple :} Start=2025, End=2030 $\rightarrow$ 6 ans = 72 mois de simulation.

\subsection{Section : Volatility Settings}

\begin{center}
\begin{tabular}{|p{4cm}|p{3cm}|p{6cm}|}
\hline
\textbf{Paramètre} & \textbf{Valeurs} & \textbf{Ce que ça change} \\
\hline
\texttt{Use Historical Volatility} & Oui / Non & Si Oui : calcule $\sigma$ depuis l'historique \\
\hline
\texttt{Volatility Multiplier} & 0.5 à 2.0 & Multiplie la volatilité historique (1.5 = 50\% plus volatile) \\
\hline
\texttt{Base Volatility (\%)} & 5\% à 50\% & Volatilité manuelle si historique désactivé \\
\hline
\end{tabular}
\end{center}

\textbf{Formule utilisée :}
$$\sigma_{\text{final}} = \sigma_{\text{historique}} \times \text{Multiplier} \times 0.6$$

Le $\times 0.6$ est un <<~dampening~>> pour éviter les courbes trop bruyantes.

\subsection{Section : Market Spikes (Black Swans)}

\begin{center}
\begin{tabular}{|p{4cm}|p{3cm}|p{6cm}|}
\hline
\textbf{Paramètre} & \textbf{Valeurs} & \textbf{Ce que ça change} \\
\hline
\texttt{Spike Frequency} & 0 à 3/an & Probabilité $p = freq/12$ par mois \\
\hline
\texttt{Spike Intensity} & 0\% à 100\% & Amplitude du spike (ex: 50\% = prix $\times$ 1.5) \\
\hline
\texttt{Spike Persistence} & 1 à 12 mois & Durée avant retour à la normale \\
\hline
\texttt{Decay Type} & Exponential / Linear & Forme de la décroissance \\
\hline
\end{tabular}
\end{center}

\textbf{Conseil pratique :}
\begin{itemize}
    \item Scénario calme : Freq=0.5, Intensity=20\%, Persistence=3
    \item Scénario stress : Freq=2.0, Intensity=50\%, Persistence=6
\end{itemize}

\subsection{Section : Simulation Settings}

\begin{center}
\begin{tabular}{|p{4cm}|p{3cm}|p{6cm}|}
\hline
\textbf{Paramètre} & \textbf{Valeurs} & \textbf{Ce que ça change} \\
\hline
\texttt{Monte Carlo Paths} & 100 à 2000 & Nombre de scénarios simulés \\
\hline
\texttt{Curve Smoothing} & 0.3 à 0.9 & Coefficient AR(1) $\alpha$ : plus haut = plus lisse \\
\hline
\texttt{Random Seed} & 0 à 99999 & Graine aléatoire (même seed = même résultat) \\
\hline
\end{tabular}
\end{center}

\subsection{Section : Scenario Explorer (Nouveau !)}

\begin{center}
\begin{tabular}{|p{4cm}|p{3cm}|p{6cm}|}
\hline
\textbf{Paramètre} & \textbf{Valeurs} & \textbf{Ce que ça change} \\
\hline
\texttt{Select Scenario} & 1 à N & Affiche le scénario \#X sur le graphique \\
\hline
\end{tabular}
\end{center}

Ce slider permet de <<~naviguer~>> dans les 500 (ou plus) scénarios simulés pour voir différentes réalisations possibles du marché.

% ============================================================================
\section{Résumé du Code Principal}
% ============================================================================

\subsection{Architecture Simplifiée}

\begin{verbatim}
app.py
  |
  +-- parse_outlook_data()      # Lit l'Excel CRU
  +-- interpolate_trend()       # Interpole en mensuel
  +-- calculate_volatility()    # Calcule sigma historique
  |
  +-- generate_brownian_motion()  # Genere le bruit lisse
  +-- generate_spike_process()    # Genere les spikes
  +-- simulate_prices()           # Combine tout
  |
  +-- create_projection_chart()   # Graphique Plotly
  +-- main()                      # Interface Streamlit
\end{verbatim}

\subsection{Fonctions Clés Expliquées}

\subsubsection{generate\_brownian\_motion()}

\begin{verbatim}
def generate_brownian_motion(n_steps, volatility, n_sims, 
                             smoothing_factor=0.7, seed=None):
    # 1) Convertir volatilite annuelle en mensuelle
    monthly_vol = (volatility / sqrt(12)) * 0.6  # Dampen
    
    # 2) Generer chocs aleatoires
    raw_shocks = normal(0, monthly_vol, (n_sims, n_steps))
    
    # 3) Appliquer lissage AR(1)
    smoothed = zeros((n_sims, n_steps))
    smoothed[:, 0] = raw_shocks[:, 0]
    for t in range(1, n_steps):
        smoothed[:, t] = (smoothing_factor * smoothed[:, t-1] + 
                          (1-smoothing_factor) * raw_shocks[:, t])
    
    # 4) Somme cumulative
    return cumsum(smoothed, axis=1)
\end{verbatim}

\subsubsection{generate\_spike\_process()}

\begin{verbatim}
def generate_spike_process(n_steps, freq, intensity, 
                           persistence, decay_type):
    # 1) Probabilite de spike par mois
    p_spike = min(freq / 12, 0.3)
    
    # 2) Tirer les spikes (bernoulli)
    spikes = binomial(1, p_spike, n_steps)
    
    # 3) Appliquer decroissance
    contribution = zeros(n_steps)
    for t in range(n_steps):
        if spikes[t] == 1:
            for k in range(persistence):
                if t + k < n_steps:
                    decay = exp(-3.0 * k / persistence)
                    contribution[t + k] += intensity * decay
    
    return contribution
\end{verbatim}

% ============================================================================
\section{Conclusion : Comment Tout S'Assemble}
% ============================================================================

\begin{enumerate}
    \item L'utilisateur choisit ses paramètres dans la sidebar
    \item On lit la tendance CRU depuis Excel
    \item On génère 500 trajectoires avec bruit + spikes
    \item On calcule moyenne et percentiles
    \item On affiche le graphique avec :
    \begin{itemize}
        \item Bande grise = zone de risque (5\%-95\%)
        \item Ligne rouge = prévision officielle
        \item Ligne bleue = scénario sélectionné
    \end{itemize}
    \item L'utilisateur peut explorer chaque scénario avec le slider
\end{enumerate}

\vspace{1cm}

\begin{center}
\fbox{\parbox{0.8\textwidth}{
\textbf{En résumé :} \\
Prix = Tendance CRU $\times$ Bruit lissé $\times$ Spikes avec décroissance \\
Répété 500 fois $\rightarrow$ Distribution de prix possibles
}}
\end{center}

\vspace{1cm}

\hrule
\vspace{0.3cm}
\small
\textbf{Document généré le 28 janvier 2026} --- Pour toute question, contacter l'équipe Data \& Analytics.

\end{document}
